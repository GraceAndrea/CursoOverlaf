%========= info ===================
% Esta es una plantilla para reportes de laboratorio
% Autor: Philip J Fry
% Licencia: Creative Commons
% ================================

\documentclass[12pt]{article} % Declarando clase de documento
%======= pre-ambulo ==========%
\usepackage[utf8]{inputenc}
% \usepackage[backend=biber, style=ieee, sorting=ynt]{biblatex}
\usepackage[backend=biber, style=apa]{biblatex}
\addbibresource{ref.bib}
%====Generador de texto Dummy===
\usepackage{lipsum}
%===== Definiendo Idioma Español =======
%\usepackage[spanish]{babel}
%\selectlanguage{spanish}
% ===Control de Margenes
\usepackage[a4paper,tmargin=3cm,bmargin=2.5cm,lmargin=3cm, rmargin=2.5cm, bindingoffset=6mm]{geometry}

%====== Datos Generales =========
\title{Paper        Conferencia}
\author{Grace A. Lema }
\date{Abril, 2020}
%======= Documento ==============
\begin{document}

\maketitle

\section{Introducción}
En \cite{nogueira2017image} se dispone de una revisión sistemática de la literatura. Sin embargo, la propuesta indicada en \cite{dawkins_biology_2016} contradicen lo habitual. El trabajo propuesto por \cite{priandana2018backprop} utiliza el algoritmo de \textit{backpropagation} para controlar un robot de rueda.\\
Este es un nuevo parrafo bla bla bla. El niño de asturias.
\vspace{1.5cm}

\lipsum[2-4]

\section{Revisión de Literatura}
\subsection{Cadena de busqueda}
\lipsum[1-4]
\subsection{Criterios de inclusión y Exclusión}
\lipsum[1-4]
\subsection{Extracción de Información}
\subsection{Criterios de Calidad}
\lipsum[1-4]
\subsection{Amenaza de la validez}
\lipsum[1-4]
\subsection{Discusión}
Aquí voy a escribir la discusión \\
\citeauthor{dawkins_biology_2016} (\citeyear{dawkins_biology_2016}) afirma que agnosticismo es una filosofía incompleto. \citeauthor{nogueira2017image} (\citeyear{nogueira2017image}) contradice la anterior afirmación y presenta un ejemplo. Finalmente, se ha demostrado que es un filosofía válida \parencite[ver pag 92]{priandana2018backprop}. "El agnosticismo indica que hay un 50\% de verdad en la tesis uno y en la tesis dos" \textcite{nogueira2017image}.

\section{Metodología}
\lipsum[1-4]
\section{Resultados Experimentales}
\lipsum[1-4]
\section{Conclusiones y trabajos Futuros}
\lipsum[1-4]
\printbibliography[title={Bibliografía}]


\end{document}
